\documentclass{article} % For LaTeX2e
% We will use NIPS submission format
\usepackage{nips13submit_e,times}
% for hyperlinks
\usepackage{hyperref}
\usepackage{url}
% For figures
\usepackage{graphicx} 
\usepackage{subfigure} 
% math packages
\usepackage{amsmath}
\usepackage{amsfonts}
\usepackage{amsopn}
\usepackage{ifthen}
\usepackage{natbib}

\title{Project-I by Group Sao Paulo}

\author{
Damien Engels\\
EPFL \\
\texttt{damien.engels@epfl.ch} \And No\'emie Jaquier\\
EPFL \\
\texttt{noemie.jaquier@epfl.ch} \\
}

% The \author macro works with any number of authors. There are two commands
% used to separate the names and addresses of multiple authors: \And and \AND.
%
% Using \And between authors leaves it to \LaTeX{} to determine where to break
% the lines. Using \AND forces a linebreak at that point. So, if \LaTeX{}
% puts 3 of 4 authors names on the first line, and the last on the second
% line, try using \AND instead of \And before the third author name.

\nipsfinalcopy 

\begin{document}

\maketitle

\begin{abstract}
In this report, ...
\end{abstract}

\section{Regression}
\subsection{Data Description}
Our regression-data contains $N=2800$ train-examples and $N=1200$ test-data. The train-data are composed of input and output variables $X$ and $y$. Each input $x_n$ has a dimensionality $D=70$ with 60 real-valued variables, 1 binary variable, 5 and 4 categorical variables with respectively 3 and 4 categories. The output of our test-data is not observed. Our goal is then to predict $y$ for all test-examples and to give an approximation of the test-error.

\subsection{Data Visualization and Cleaning}
To visualise our data, we plotted histograms to see the distribution of $y$ and of $X$ dimensions. The input data is not centred, so that we need to normalise it. 

Figure \todo{ADD FIGURE} shows the histogram of the output variable. We see that the data are split in three bursts of different sizes : a large amount of the data have small output values and only a small part have the highest values. Figure \todo{ADD FIGURE} shows histograms of two interesting dimensions of $X$. As all others real-valued variables seem to have sort of Gaussian distributions, we can see that those are split in two bursts. By plotting $y$ in function of each of these variables, as shown in Figure \todo{ADD FIGURE}, we can see that $y$ is highly correlated with dimensions $6$ and $55$ of $X$. The output-data can thus be separated in three clusters by looking at $x_n(6)$ ans $x_n(55)$ for each data.

We also computed the eigenvalues of the matrix $X^T X$. As ten eigenvalues are smaller than $10^{-10}$, we will need to transform $X$ to remove dimensions corresponding to those small eigenvalues or to use methods such as ridge regression to lift them. 

%\begin{figure}[!h] % !t
%\center
%\subfigure[Boxplot of real-valued $\mathbf{X}$. Data is not centered and therefore we normalize it.]{\includegraphics[width=2.5in]{figures/boxplotX.pdf} \label{fig:boxplotX}}
%\hfill
%\subfigure[Histogram of $\mathbf{y}$. We can clearly see two outliers.]{\includegraphics[width=2.5in]{figures/histY.pdf} \label{fig:histY}}
%\caption{}
%\end{figure}

\subsection{Clustering}
We used k-means clustering to separate our data during the training part. The clustering is always done before normalising the data.
As the data are not equally shared between the three clusters, we first compute a clustering with $k=2$ clusters using $y$ and the most correlated dimension of $X$ to compute the distance between each data-point and centre. Then, we keep only the data classified in the sparse cluster to make a second clustering with $k=2$ using $y$ and the second most correlated dimension of $X$.

In order to ensure a reliable clustering, we do not initialise the centres randomly. During the initialisation, we sort the data-points by their norm and separate the sorted list in $k$ chunks, where $k$ is the given number of clusters. The mean along each dimension is then calculated for each chunk giving the initial centres for the clusters.

During the test part, we follow the same principle. The most correlated dimension of $X$ found during the training part is used to find the closest centre and the second most correlated dimension is then used to divide the data classified in the sparse cluster in two sub-clusters.

Figure \todo{ADD FIGURE} shows an example of k-means clustering.

\subsection{Feature transformations}
After the separation in cluster, each group of train-data is treated separately. For each cluster, the dimensions of $X$ and $y$ are normalised to have zero-mean and unitary standard deviation. We also implemented a method to fix the outliers and a PCA to improve our predictions. 

We defined as outliers the data-points for which some elements of one or more real-valued dimensions are further than 3 standard deviations from the mean. As our data were normalised, all non-categorical element higher than 3 were considered as an outlier element. To fix the outliers, value $3$ (or $-3$) was assigned to those elements.

The PCA is applied only on the real-valued dimensions. It projects our input data in a new coordinate system based on the covariance between dimensions. It allows us to reduce the dimensionality of our data in this new space by removing the dimensions with eigenvalues lower than 1. The data are then projected back in the initial coordinate system. In our case, the PCA reduce our input data by eleven dimensions.
\todo{ADD FIGURE WITH EIGENVALUES}

After the clustering of test-data, each group is also treated separately. The data are first normalised given the means and standard deviations computed during the training. The outliers are then fixed and PCA is applied by projecting the data with the eigenvectors computed during the training part.

\subsection{Methods}
We applied least-squares and ridge regression to this dataset... (remove non-real-valued dimensions)

\section{Classification}
\subsection{Data Description}
Our classification-data contains $N=1500$ train-examples and $N=1500$ test-data. The train-data are composed of input and output variables $X$ and $y$. Each input $x_n$ has a dimensionality $D=37$ with 32 real-valued variables, 2 binary variables, 3 categorical variables with 5 categories. The output is a binary variable equal to $1$ or $-1$. The output of our test-data is not observed. Our goal is then to predict $y$ for all test-examples and to give an approximation of the test-error.

\subsection{Data Visualization and Cleaning}
To visualise our data, we plotted histogram of $X$ dimensions. The input data is again not centred, so that we need to normalise it.

Figure \todo{ADD FIGURE} shows the histogram of the \todo{DIMENSION} variable of $X$. We see that the distribution is composed of a high peak and a Gaussian distribution. We observed that most of the real-valued dimensions have similar distributions and that data-points belonging to the peak or to the Gaussian in one dimensions are contained in the same part of the distribution in other dimensions. Moreover, the binary value of the 20th dimension is different for data-points in the peak or in the Gaussian, so that we can use this dimension to separate the input data in two clusters.

We also computed the eigenvalues of the matrix $X^{T}X$. As the smaller eigenvalue is equal to $58.4$, the matrix is full-rank and we do not need to use methods to lift the eigenvalues.

\subsection{Feature Transformations}
The train-data are first separated in two clusters given the value of the 20th dimension of $X$. Each group of data are then treated separately. $X$ real-valued variables are first normalised to have zero-mean and unitary standard deviation. We also implemented a method to remove the outliers.

We defined as outliers the data-points for which at least one element of a real-valued dimension is further than 3 standard deviations from the mean. In this case, the data point is removed from the training set. 

After the clustering of the test-data, each group is also treated separately. The real-valued input data are normalised given the means and standard deviations computed during the training.

\subsection{Logistic Regression}


\section{Summary}
In this report, we ...


\subsubsection*{Acknowledgments}
...

\subsubsection*{References}

\end{document}
